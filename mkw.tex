\documentclass[sigplan,balance,pbalance,natbib=false]{acmart}
%% ∞
%% …
%% ₀
%% ₁
%% ₂
%% ₃
\usepackage{fontspec,newunicodechar}
\usepackage{hyphenat} %% electromagnetic\hyp{}endioscopy
\usepackage[shortcuts]{extdash}
\usepackage{microtype} %% provides tighter text formatting
\usepackage{flushend}
\usepackage{balance}
\usepackage{polyglossia} %% for xelatex use polyglossia, needed for csquotes
\setdefaultlanguage[variant=american]{english}
\usepackage{fancybox}

\PassOptionsToPackage{cache=false,newfloat=true}{minted}
\PassOptionsToPackage{unicode}{hyperref}

\usepackage[capitalize,nameinlink]{cleveref} %% adds \crefrange{}{} and \cpagerefrange{}
%% \BibTeX command to typeset BibTeX logo in the docs
\AtBeginDocument{%
  \providecommand\BibTeX{{%
    Bib\TeX}}}

\RequirePackage[
  datamodel=acmdatamodel,
%%  style=acmnumeric,
  ]{biblatex}

%% Declare bibliography sources (one \addbibresource command per source)
\addbibresource{mkw.bib}

\overfullrule=0pt

\usepackage{nag} %% should complain about old and outdated commands
\usepackage{minted}
\usemintedstyle{bw}

\usepackage[strict,autostyle]{csquotes} %% enquote &c that doesn't interfere w/cdlatex usage
%% http://ftp.lyx.org/pub/tex-archive/macros/latex/contrib/csquotes/csquotes.pdf
%% \usepackage[xspace]{ellipsis} they suggest using xetex instead
%% \usepackage{setspace} %% set the spacing (single double, etc)

\setmonofont{DejaVuSansMono}[Scale=MatchLowercase]
\tracinglostchars=2
\usepackage[defaultlines=3,all]{nowidow} %% Eliminates widow and orphans
\setlength {\marginparwidth }{2cm}
\usepackage{todonotes}

\newmintinline[rackinline]{racket}{}
\newmintinline{prolog}{}

\setcopyright{acmcopyright}
\copyrightyear{2023}
\acmYear{2023}
\acmDOI{XXXXXXX.XXXXXXX}

\acmConference[TFP '23]{Symposium on Trends in Functional
  Programming}{January 13-15, 2023}{Boston, Massachusetts}

\acmPrice{15.00}
\acmISBN{978-1-4503-XXXX-X/18/06}

\begin{document}

\title{Nearly Macro-free microKanren}

\author{Jason Hemann}
\email{jason.hemann@shu.edu}
\orcid{0000-0002-5405-2936}
\author{Daniel P. Friedman}
\email{dfried@indiana.edu}
\orcid{0000-0001-9992-1675}

\renewcommand{\shortauthors}{Hemann and Friedman}

\begin{abstract}

  We describe changes to the microKanren implementation that make it
  more practical to use in a host language without macros. With the
  help of some modest runtime features common to most languages, we
  show how an implementer lacking macros can come closer to the
  expressive power that macros usually provide---with varying degrees
  of success. The result is a still functional microKanren that
  invites slightly shorter programs, and is relevant even to
  implementers that enjoy macro support. For those without it, we
  address some pragmatic concerns that necessarily occur without
  macros so they can better weigh their options.

\end{abstract}

\begin{CCSXML}
<ccs2012>
   <concept>
       <concept_id>10011007.10011006.10011008.10011009.10011015</concept_id>
       <concept_desc>Software and its engineering~Constraint and logic languages</concept_desc>
       <concept_significance>500</concept_significance>
   </concept>
   <concept>
       <concept_id>10003752.10003790.10003795</concept_id>
       <concept_desc>Theory of computation~Constraint and logic programming</concept_desc>
       <concept_significance>300</concept_significance>
   </concept>
 </ccs2012>
\end{CCSXML}

\ccsdesc[500]{Software and its engineering~Constraint and logic languages}

%%
 %% Keywords. The author(s) should pick words that accurately describe
%% the work being presented. Separate the keywords with commas.
\keywords{logic programming, miniKanren, DSLs, embedding, macros}

\maketitle

\section{Introduction}

Initially we designed microKanren~\cite{hemann2013muKanren} as a
compact relational programming language kernel to undergird a
miniKanren implementation. Macros are used to implement the
surrounding higher-level miniKanren operators and surface syntax.\@
microKanren is often used as a tool for understanding the guts of a
relational programming language through studying its implementation.
By re-implementing miniKanren as separate surface syntax macros over a
purely-function microKanren kernel, we hoped that this separation
would simultaneously aid implementers when studying the source code,
and also that the functional core would make the language easier to
port to other functional hosts. To support both of those efforts, we
also chose to program in a deliberately small and workaday set of
Scheme primitives.

The sum of those implementation restrictions, however, seemed to force
some awkward choices including binary logical operators, one at a time
local variable introduction, and leaks in the stream abstractions.
These made the surface syntax macros seem almost required, and were
far enough from our goals that \emph{The Reasoned Schemer, 2nd
  Ed}~\cite{friedman2018reasoned} did not use a purely functional
kernel. It also divided host languages into those that used macros and
those that did not. We bridge some of that split by re-implementing
parts of the kernel using some modest runtime features common to many
languages.

Here we:
%
\begin{itemize}

\item show how to functionally implement more general logical
  operators, cleanly obviating some of the surface macros,

\item survey the design space of purely functional implementation
  alternatives for the remaining macros in the \emph{TRS2e} core
  language implementation, and weigh the trade-offs and real-world
  consequences, and

\item suggest practical solutions for completely eliminating the
  macros in those places where the pure microKanren functional
  implementations seemed impractical.

\end{itemize}

This resulted in some higher-level (variadic rather than just binary)
operators, a more succinct kernel language, and enabled some
performance improvement. Around half of the changes are applicable to
any microKanren implementation, and the more concise goal combinators
of~\cref{sec:conde} may also be of interest to implementers who embed
goal-oriented languages like Icon~\cite{griswold1983icon}. The other
half are necessarily awkward yet practical strategies for those
platforms lacking macro support. The source code for both our
re-implementation and  our experimental results is available
at~\url{https://github.com/jasonhemann/tfp-2023/}.

In \cref{sec:all-aboard}, we illustrate by example what made surface
syntax macros feel practically mandatory. In \cref{sec:conde}, we
implement conjunction and disjunction, and in \cref{sec:impure} we
discuss the re-implementation of the impure operators. We discuss the
remaining macros in \cref{sec:functional}. We close with some
outstanding questions on performance impacts of these implementation
choices, and consider how Kanren language implementers outside of the
Scheme family might benefit from these alternatives.

\section{Limitations of microKanren}\label{sec:all-aboard}

We assume the reader is familiar with the miniKanren implementation of
\emph{TRS2e}. Although based on microKanren, this implementation makes
some concessions to efficiency and safety and uses a few macros in the
language kernel itself. In addition to that implementation, in this
paper we make occasional references to earlier iterations such as
\citeauthor{hemann2016small}~\cite{hemann2016small}, an expanded
archival version of the 2013 paper~\cite{hemann2013muKanren}.

The Carmelit in Haifa is the world's shortest subway system with only
six stations on its line: an example sufficiently small that modeling
it should be painless. But in microKanren, to model the order a
passenger travels past the stops riding that subway end to end
requires 11 logical operator nodes, because microKanren only provides
\emph{binary} conjunctions and disjunctions. (\Cref{mnt:new-carmelit}
contains this paper's alternative solution, requiring just 3.) For a
logic programming language, solely binary logical operators is too low
level. In our view, this makes the superficial syntax macros
practically mandatory, and impedes host languages without a macro
system.

Moreover, the microKanren language doesn't offer the programmer
sufficient guidance in using that fine-grained control. With $n$
goals, the programmer can associate to the left, to the right, or some
mixtures of the two. The syntax does not obviously encourage any one
choice. Subtle changes in program structure can have profound effects
on performance, and mistakes are easy to make.

Similarly, the soft-cut operator \rackinline{ifte} in the \emph{TRS2e}
language kernel is also low level. It permits a single test, a single
consequent, and a single alternative. To construct an if-then-else
cascade, a microKanren programmer without the \rackinline{conda}
surface macro would need to code that unrolled conditional expression
by hand.

The core \emph{TRS2e} language implementation relies on
macros \rackinline{fresh}, \rackinline{defrel}, and \rackinline{run}
to introduce new logic variables, globally define relations, and
execute queries. Earlier implementations of those same behaviors via
pure functional shallow embeddings, without macros, had some harsh
consequences. We will revisit those earlier implementations and their
trade-offs, survey the landscape of available choices, and suggest
compromises for those truly without macros, thus increasing
microKanren's \emph{practical} portability.

\section{The \rackinline{disj} and \rackinline{conj} goal
  constructors}\label{sec:conde}

microKanren's binary \rackinline{disj₂} and \rackinline{conj₂}
operators, shown in \cref{mnt:disj2-conj2}, are goal combinators: they
each take two goals, and produce a new goal. Disjunction and
conjunction work slightly differently. The program attempts to achieve
a \emph{goal}: it can fail or succeed (and it can succeed multiple
times). A goal executes with respect to a \emph{state}, here the
curried parameter \rackinline{s}, and the result is a \emph{stream} of
states, usually denoted \rackinline{s∞} as each entry is a state that
results from achieving that goal in the given state.
The
\rackinline{append∞} function used in \rackinline{disj₂} is a kernel
primitive that combines two streams into one, with an interleave
mechanism to prevent starvation; the result is a stream of the ways to
achieve the two goals' disjunction. The \rackinline{append-map∞}
function used in \rackinline{conj₂} is to
\rackinline{append∞} what \rackinline{append-map} is
to \rackinline{append}. The ways to achieve the conjunction of two
goals are all the ways to achieve the second goal in a state that
results from achieving the first goal.\@ \rackinline{append-map∞} runs
the second goal over the stream of results from the first goal, and
combines together the results of mapping into a single stream. That
stream represents the conjunction of the two goals, again with special
attention to interleaving and starvation.

\begin{listing}
  \begin{minted}[autogobble,stripall]{racket}
(define ((disj₂ g₁ g₂) s)
  (append∞ (g₁ s) (g₂ s)))

(define ((conj₂ g₁ g₂) s)
  (append-map∞ g₂ (g₁ s)))
  \end{minted}
  \caption{microKanren \rackinline{disj₂} and \rackinline{conj₂}}\label{mnt:disj2-conj2}
\end{listing}

We want to implement as functions disjunction and conjunction over
arbitrary quantities of goals. These implementations should subsume
the binary \rackinline{disj₂} and \rackinline{conj₂} and they also
should not use \rackinline{apply}. Further, they should not build
extraneous closures: unnecessarily building closures at runtime is
always a bad idea. This re-implementation requires a host that
supports variable arity functions, a widely available feature included
in such languages as JavaScript, Ruby, Java, and Python. These
languages do not generally support macros and hence can use these ideas.

\begin{listing}
\begin{minted}[autogobble,stripall,frame=single]{racket}
  (define ((disj . gs) s)
    (cond
      ((null? gs) (list))
      (else (D ((car gs) s) (cdr gs) s))))

  (define (D s∞ gs s)
    (cond
      ((null? gs) s∞)
      (else
       (append∞ s∞
         (D ((car gs) s) (cdr gs) s)))))

  (define ((conj . gs) s)
    (cond
      ((null? gs) (list s))
      (else (C (cdr gs) ((car gs) s)))))

  (define (C gs s∞)
    (cond
      ((null? gs) s∞)
      (else
       (C (cdr gs)
          (append-map∞ (car gs) s∞)))))
\end{minted}
  \caption{Final re-definitions of \rackinline{disj} and \rackinline{conj}}\label{mnt:disj-reimplementation}
\end{listing}


\Cref{mnt:disj-reimplementation} shows our new versions, implemented
as shallow wrappers over simple folds. The first steps are to dispense
with the trivial case, and then to call a recursive help function that
makes no use of variadic functions. The focus is on recurring over the
list \rackinline{gs}. Unlike \rackinline{D}, the
function \rackinline{C} does not take in the state \rackinline{s}; the
help function does not need the state for conjunction. In each
recursive call, we accumulate by mapping the next goal in the list
using that special delaying implementation of \rackinline{append-map∞}
for Kanren-language streams. This left-fold implementation of
conjunctions therefore left-associates the conjuncts.

\subsection{Deriving semantic equivalents}

A developer might derive these definitions as follows. We start with
the definition of a recursive \rackinline{disj} macro like one might
define as surface syntax over the microKanren \rackinline{disj₂}. As
this is not part of the microKanren language, we would like to
dispense with the macro and implement this behavior functionally. At
the cost of an \rackinline{apply}, we can build the corresponding
explicitly recursive \rackinline{disj} function.
Since \rackinline{disj} produces and consumes goals, we can η-expand
that first functional definition by a curried
parameter \rackinline{s}. We then split \rackinline{disj} into two
mutually-recursive functions, the third and fourth definitions in
\cref{mnt:disj-substituted-through}.

We can replace the call to \rackinline{disj₂} in that version by its
definition in terms of \rackinline{append∞} and perform a trivial
β-reduction. The explicit \rackinline{s} argument suggests removing
the call to \rackinline{apply} and making \rackinline{D} recursive.
The result is the final version of \rackinline{D} in
\cref{mnt:disj-substituted-through}. The definition
of \rackinline{disj} remains unchanged from before. In both clauses
of \rackinline{D} we combine \rackinline{g} and \rackinline{s}, this
suggests constructing that stream in \rackinline{disj} and passing it
along. Adding the trivial base case to that \rackinline{disj} yields
the definition in~\cref{mnt:disj-reimplementation}.

\begin{listing}
\begin{minted}[autogobble,stripall]{racket}
(define-syntax disj
  (syntax-rules ()
    ((disj g) g)
    ((disj g₀ g₁ g ...)
     (disj₂ g₀ (disj g₁ g ...)))))

(define (disj g . gs)
  (cond
    ((null? gs) g)
    (else (disj₂ g (apply disj gs)))))

(define ((disj g . gs) s)
  (D g gs s))

(define (D g gs s)
  (cond
    ((null? gs) (g s))
    (else ((disj₂ g (apply disj gs)) s))))

(define (D g gs s)
  (cond
    ((null? gs) (g s))
    (else
      (append∞ (g s)
        (D (car gs) (cdr gs) s)))))
\end{minted}
  \caption{Derivation of \rackinline{disj} function definition}\label{mnt:disj-substituted-through}
\end{listing}

We can derive the definition of \rackinline{conj} from
\cref{mnt:disj-reimplementation} via a similar process. Starting with
the variadic function based on the macro in
\cref{mnt:conj-substituted-through}, we first η-expand and split the
definition. We next substitute for the definitions
of \rackinline{conj} and \rackinline{conj₂}. Finally,
since \rackinline{C} only needs \rackinline{s} to \emph{build} the
stream, we can assemble the stream on the way in---instead of passing
in \rackinline{g} and \rackinline{s} separately, we pass in their
combination as a stream. The function is tail recursive, we can change
the signature in the one and only external call and the recursive
call. The result, after adding the trivial base case
to \rackinline{conj}, is shown in \cref{mnt:disj-reimplementation}.

Both the functional and the macro based versions of
\cref{mnt:conj-substituted-through} use a left fold over the goals,
whereas the versions of \rackinline{disj} use a right fold. This is
not an accident.
%
Folklore suggests left associating conjunctions tends to improve the
performance of miniKanren's interleaving search. The authors know of
no thorough algorithmic proof of such claims, but see for instance
discussions and implementation in~\cite{rosenblatt2019first} for some
of the related work so far. We have generally, however, resorted to
small step visualizations of the search tree to explain the
performance impact. It is worth considering if we can make an equally
compelling argument for this preference through equational reasoning
and comparing the implementations of functions.

% Folklore suggests that left associating conjunctions tends to improve
% the performance of miniKanren's interleaving search. The authors know
% of no thorough algorithmic proof of such claims, but see for instance
% discussions and implementation in~\cite{rosenblatt2019first} for some
% of the related work so far. In \cref{tab:???}, we display the results
% of some micro benchmarks that suggest the same. We have generally,
% however, resorted to small step visualizations of the search tree to
% explain the performance impact. The authors believe it is worth
% considering if we can make an equally compelling argument for this
% preference through equational reasoning and comparing the
% implementations of functions.

The benefits of a left-fold over conjunctions becomes a little more
obvious by comparison to a right-fold implementation after we
η-expand, unfold to a recursive help function, substitute in the
definition of \rackinline{conj₂}, and β-reduce. From there, we cannot
(easily) replace the \rackinline{apply} call by a recursive call
to \rackinline{C}, because we are still waiting for an \rackinline{s}.
We can only abstract over \rackinline{s} and wait; we show the upshot
of this sequence in \Cref{mnt:conj-right-fold-definition}. The
equivalent right-fold implementation needs to somehow construct a
closure for every recursive call. Building these superfluous closures
is expensive. The same closure stacking behavior appears in the right
fold variant of \rackinline{disj}. Basic programming horse sense
suggests the more elegant variants from
\cref{mnt:disj-reimplementation}.

\begin{listing}[h]
\begin{minted}[autogobble,stripall]{racket}
(define-syntax conj
  (syntax-rules ()
    ((conj g) g)
    ((conj g g₁ gs ...)
     (conj (conj₂ g g₁) gs ...))))

(define (conj g . gs)
  (cond
    ((null? gs) g)
    (else
     (apply conj
       (cons (conj₂ g (car gs)) (cdr gs))))))

  (define ((conj g . gs) s)
    (C g gs s))

  (define (C g gs s)
    (cond
      ((null? gs) (g s))
      (else
       ((apply conj
          (cons (conj₂ g (car gs)) (cdr gs)))
        s))))

  (define (C g gs s)
    (cond
      ((null? gs) (g s))
      (else
       (C (λ (s)
            (append-map∞ (car gs) (g s)))
          (cdr gs)
          s))))
\end{minted}
  \caption{Derivation of split \rackinline{conj} function definition}\label{mnt:conj-substituted-through}
\end{listing}

\begin{listing}[h]
\begin{minted}[autogobble,stripall]{racket}
  (define ((conj g . gs) s)
    (C gs (g s)))

  (define (C gs s∞)
    (cond
      ((null? gs) s∞)
      (else
       (append-map∞
         (λ (s)
           (C (cdr gs) ((car gs) s)))
         s∞))))
\end{minted}
  \caption{A right-fold variant of \rackinline{conj} after some derivations}\label{mnt:conj-right-fold-definition}
\end{listing}

The new \rackinline{disj} and \rackinline{conj} functions are, we
believe, sufficiently high-level for programmers in implementations
without macros. Though this note mainly concerns working towards an
internal macro-less kernel language, it may also have something to say
about the miniKanren-level surface syntax, namely that even the
miniKanren language could do without its \rackinline{conde} syntax (a
disjunction of conjunctions that looks superficially like
Scheme's \rackinline{cond}) and have the programmer use these new
underlying logical primitives. In \cref{mnt:new-carmelit}, we
implement \rackinline{carmelit-subway} as an example, and it reads
much better than the 11 binary logical operators the programmer would
have needed in the earlier version.


\begin{listing}[h]
  \begin{minted}[autogobble,stripall]{racket}
(defrel (carmelit-subway a b c d e f)
  (disj
    (conj (== a 'carmel-center)
          (== b 'golomb)
          (== c 'masada)
          (== d 'haneviim)
          (== e 'hadar-city-hall)
          (== f 'downtown))
    (conj (== a 'downtown)
          (== b 'hadar-city-hall)
          (== c 'haneviim)
          (== d 'masada)
          (== e 'golomb)
          (== f 'carmel-center))))
  \end{minted}
  \caption{A new Carmelit subway without \rackinline{conde}}\label{mnt:new-carmelit}
\end{listing}

\section{Tidying up the impure operators}\label{sec:impure}

The \rackinline{conda} of \emph{TRS2e} provides nested
\enquote{if-then-else} behavior. It relies on microKanren's
underlying \rackinline{ifte}. The definition of \rackinline{conda}
requires one or more conjuncts per clause and one or more clauses. The
last line of \rackinline{conda} contains the only place in the
implementation that relies structurally on permitting nullary
conjunctions, or disjunctions, of goals. Everywhere else conjunctions
are one-or-more, and this one structural dependency is off-putting.
There is a temptation to rewrite the second pattern in
miniKanren's \rackinline{conda} to demand \emph{two} or more goals in
each if-then clause and remove the dependency.

% (define-syntax conda
%   (syntax-rules ()
%     ((conda (g₀ g ...)) (conj g₀ g ...))
%     ((conda (g₀ g ...) ln ...)
%      (ifte g₀ (conj g ...) (conda ln ...)))))

% (define ((ifte g₁ g₂ g₃) s)
%   (let loop ((s∞ (g₁ s)))
%     (cond
%       ((null? s∞) (g₃ s))
%       ((pair? s∞)
%        (append-map∞ g₂ s∞))
%       (else (lambda ()
%               (loop (s∞)))))))

% (define ((conda q a . q-and-a*) s)
%   (A (q s) a q-and-a* s))

% (define (A s∞ a q-and-a* s)
%   (cond
%     ((null? s∞)
%      (cond
%        ((null? (cdr q-and-a*)) ((car q-and-a*) s))
%        (else (A ((car q-and-a*) s)
%                 (cadr q-and-a*)
%                 (cddr q-and-a*)
%                 s))))
%     ((pair? s∞) (append-map∞ a s∞))
%     (else (lambda () (A (s∞) a q-and-a* s)))))

Some microKanren programmers without macros would be perfectly
satisfied just using \rackinline{ifte} directly, especially so given
the research community's focus on purely relational programming. But
just as the standard forked \rackinline{if} led to
McCarthy's \rackinline{if} notation and \rackinline{cond}, a
programmer may eventually feel the need for a nested implementation.
The lower portion of \cref{mnt:conda-implementation} contains a
functional implementation of that cascade behavior. It includes the
delay-and-restart behavior of \rackinline{ifte} together
with \rackinline{conda}'s logical cascade. The \rackinline{s∞} can be
either empty, non-empty, or a function of no arguments. In the last
case, we invoke \rackinline{s∞}. Rather than building a largely
redundant implementation of \rackinline{condu}, we expose the
higher-order goal \rackinline{once} to the user. We take the
definition of \rackinline{once} directly from \emph{TRS2e}. The
programmer can simulate \rackinline{condu} by
wrapping \rackinline{once} around every test goal.


\begin{listing}[h]
  \begin{minted}[autogobble,stripall]{racket}
(define (conda . <qa>*e)
  (A <qa>*e))

(define (A <qa>*e)
  (B (car <qa>*e) (cadr <qa>*e) (cddr <qa>*e)))

(define (B q a <qa>*e)
  (cond
    ((null? (cdr <qa>*e)) (ifte q a (car <qa>*e)))
    (else (ifte q a (A <qa>*e)))))

(define ((ifte g₁ g₂ g₃) s)
  (let loop ((s∞ (g₁ s)))
    (cond
      ((null? s∞) (g₃ s))
      ((pair? s∞) (append-map∞ g₂ s∞))
      (else (lambda () (loop (s∞)))))))

(define ((once g) s)
  (let loop ((s∞ (g s)))
    (cond
      ((null? s∞) '())
      ((pair? s∞) (cons (car s∞) '()))
      (else (lambda () (loop (s∞)))))))
  \end{minted}
  \caption{A functional \rackinline{conda}, \rackinline{ifte}, and \rackinline{once}}\label{mnt:conda-implementation}
\end{listing}

\section{Remainders and practicalities}\label{sec:functional}

The \citeyear{hemann2013muKanren} microKanren paper demonstrates how
to implement a purely functional Kanren language in an eager host. In
\cref{mnt:call-fresh-and-call-initial-state} we display these
alternative mechanisms for introducing fresh logic variables,
executing queries, and introducing delay and interleave. The versions
in \cref{mnt:call-fresh-and-call-initial-state} are slightly adjusted
to be consistent with this presentation.

\begin{listing}
  \begin{minted}[autogobble,stripall]{racket}
(define ((call/fresh f) s)
  (let ((v (state->newvar s)))
    ((f v) (incr-var-ct s))))

(define (call/initial-state n g)
  (reify/1st
    (take n (pull (g init-state)))))

(define (((Zzz g) s))
  (g s))
  \end{minted}
  \caption{Functional microKanren equivalents of \emph{TRS2e} kernel macros}\label{mnt:call-fresh-and-call-initial-state}
\end{listing}

Each of these has drawbacks that compelled the \emph{TRS2e} authors to
instead use macro-based alternatives in the kernel layer. In this
section, we explicitly address those drawbacks and point out some
other non-macro alternatives that may demand more from a host language
than the original microKanren choices, and offer some thoughts on the
choice.

\paragraph{Logic variables}

Many of the choices for these last option hinge on a representation of
logic variables. Every implementation must have a mechanism to produce
the next fresh logic variable. The choice of variable representation
will affect the implementation of unification and constraint solving,
the actual introduction of fresh variables, as well as answer
projection, the formatting and presentation of a query's results.
Depending on the implementation of variables, you may also need
additional functions to support your implementation of variables. In a
shallow embedding, designating some set as logic variables means
either using a \rackinline{struct}-like mechanism to construct a
bespoke datatype hidden from the end user, or to arrogate some subset
of the host language's values for use as logic variables.
Using \rackinline{struct}s and limiting the visibility of the
constructors and accessors is a nice option for languages that support
it.

The choice of which host language values to take for logic variables
divides roughly into the structurally equal and the referentially
equal. For an example of the latter, consider representing each
variable using a vector, and identifying vectors by their unique
memory location. This latter approach models logic variables as a
single global pool rather than reused separately across each disjunct,
and so requires more logic variables overall. The microKanren approach
implicitly removes numbers, as data, from the user's term language and
uses the natural numbers directly as an indexed set of variables.

\paragraph{\rackinline{fresh}}

There are numerous ways to represent variables, and so too are there
many ways to introduce fresh variables. In the microKanren approach,
the current variable index is one of the fields of the state threaded
through the computation; to go from index to variable is the identity
function, and the \rackinline{state->newvar} we use can be just an
accessor. The function \rackinline{incr-var-ct} can reconstruct that
state with the variable count incremented. The \rackinline{call/fresh}
function takes as its first argument, a goal parameterized by the name
of a fresh variable. \rackinline{call/fresh} then applies that
function with the newly created logic variable, thereby associating
that host-language lexical variable with the DSL's logic variable.
This lets the logic language \enquote{piggyback} on the host's lexical
scoping and variable lookup.

This approach also means, however, that absent some additional
machinery the user must introduce those new logic variables one at a
time, once each per \rackinline{call/fresh} expression, as though
manually currying all functions in a functional language. This made
programs larger than the relational \rackinline{append} difficult to
write and to read, and that amount of threading and re-threading state
for each variable is costly. We could easily support, say instead,
three variables at a time---force the user to provide a three-argument
function and always supply three fresh variables at a time. Though
practically workable the choice of some arbitrary quantity $k$ of
variables at a time, or choices $k_{1}$ and $k_{2}$ for that matter,
seems unsatisfactory. It could make sense to inspect a procedure for
its arity at runtime and introduce exactly that many variables, in
languages that support that ability. In many languages, a procedure's
arity is more complex than a single number. Variadic functions and
keyword arguments all complicate the story of a procedure's arity. A
form like \rackinline{case-lambda} means that a single procedure may
have several disjoint arities. The arity inspection approach could be
a partial solution where the implementer restricts the programmer to
using functions with fixed arity.

One last approach is to directly expose to the user a mechanism to
create a new variable, and allow the programmer to use something like
a \rackinline{let} binding to do their own variable introduction and
name binding. Under any referentially transparent representation of
variables, this would mean that the programmer would be responsible
for tracking the next lexical variable. This last approach pairs best
with referentially opaque variables where the operation to produce a
new variable allocates some formerly unused memory location so the
programmer does not need to track the next logic variable. See
sokuza-kanren~\cite{kiselyov2006taste} for an example of this style.
With this latter approach, however, we can expose \rackinline{var}
directly to the programmer and the programmer can use \rackinline{let}
bindings to introduce several logic variables simultaneously.

\paragraph{\rackinline{run}}

\Cref{mnt:call-fresh-and-call-initial-state} also shows how we have
implemented a \rackinline{run}-like behavior without using macros.
Using the purely-functional implementation of logic variables, a
function like \rackinline{call/initial-state} can do the job
of \rackinline{run} and \rackinline{run*}. The query is itself
expressed as a goal that introduces the first logic
variable \rackinline{q}. A run-like operator displays the result with
respect to the first variable introduced. This means pruning
superfluous variables from the answer, producing a single value from
the accumulated equations, and numbering the fresh variables. When
logic variables are only identified by reference equality, the
language implementation must pass \emph{the} actual same logic
variable into the query, and also to the answer projection,
called \rackinline{reify}. The pointer-based logic variable approach
forces the programmer to explicitly invoke \rackinline{reify} as
though it were a goal as the last step of executing the query, as in
\cref{mnt:run-query}, or create a special variable introduction
mechanism for the first variable, scoped over both the query and the
answer projection.

\begin{listing}
  \begin{minted}[autogobble,stripall]{racket}
(call/initial-state 1
  (let ((q (var 'q)))
    (conj
      (let ((x (var 'x)))
        (== q x))
      (reify q))))
  \end{minted}
  \caption{Queries as expressed with global-state variables}\label{mnt:run-query}
\end{listing}

\paragraph{\rackinline{define}}

The microKanren programmer can just use their host
language's \rackinline{define} feature to construct relations as
host-language functions, and manually introduce the delays in
relations using a help function like \rackinline{Zzz} to introduce
delays, as in the original implementation.~\cite{hemann2013muKanren}
This may be a larger concession than it looks, since it exposes the
delay and interleave mechanism to the user, and both correct
interleaving and, in an eager host language, even the termination of
relation \emph{definitions} rely on a whole-program correctness
property of relation definitions having a delay. \rackinline{Zzz} if
\emph{always used correctly} would be sufficient to address that
problem, but forgetting it just once would cause the entire program to
loop. Turning the delaying and interleaving into a user-level
operation means giving the programmer some explicit control over the
search, and that in turn could transform a logic language into an
imperative one. Another downside of relying on a
host-language \rackinline{define} is that the programmer must now take
extra care not to provide multiple goals in the body.
The \rackinline{define} form will treat all but the last expression as
statements and silently drop them, rather than conjoin them as
in \rackinline{defrel}. That can be a subtle mistake to debug. Those
implementing a shallowly embedded stream-based implementation in an
eager host language may, however, have little choice.

\section{Future work}\label{sec:conclusion}

This note shows how to provide a somewhat more concise core language
that significantly reduces the need for macros, and provides some
alternatives for working without macros that may be more practical
than those of~\citeauthor{hemann2013muKanren}.

Forcing ourselves to define \rackinline{disj} and \rackinline{conj}
functionally, and with the restrictions we placed on ourselves in this
re-implementation, removed a degree of implementation freedom and led
us to what seems like the right solution. The result is closer to the
design of Prolog, where the user represents conjunction of goals in
the body of a clause with a comma and disjunction, either implicitly
in listing various clauses or explicitly with a semicolon. The prior
desugaring macros do not seem to suggest how to associate the calls to
the binary primitives---both left and right look equally nice---where
these transformations suggest a reason for the performance difference.

Existing techniques for
implementing \rackinline{defrel}, \rackinline{fresh}
and \rackinline{run} (and \rackinline{run*}) without macros have with
serious drawbacks. These include exposing the implementation of
streams and delays, and the inefficiency and clumsiness of introducing
variables one at a time, or the need to reason with global state. With
a few more runtime features from the host language, an implementer can
overcome some of those drawbacks, and may find one the suggested
solutions an acceptable trade-off.

From time to time we find that the usual miniKanren implementation is
\emph{itself} lower-level than we would like to program with
relations. Early microKanren implementations restrict themselves
to \rackinline{syntax-rules} macros. Some programmers use macros to
extend the language further as
with \rackinline{matche}~\cite{keep2009pattern}. Some constructions
over miniKanren, such
as \rackinline{minikanren-ee}~\cite{ballantyne2020macros}, may rely on
more expressive macro systems
like \rackinline{syntax-parse}~\cite{culpepper2012fortifying}.

We would still like to know if our desiderata here are \emph{causally}
related to good miniKanren performance. Can we reason at the
implementation level and peer through to the implications for
performance? If left associating \rackinline{conj} is indeed uniformly
a dramatic improvement, the community might consider reclassifying
left-associative conjunction as a matter of correctness rather than an
optimization, as in \enquote{tail call optimization} vs.
\enquote{Properly Implemented Tail Call
  Handling}~\cite{felleisen2014requestions}. Regardless, we hope this
document helps narrow the gap between implementations in functional
host languages with and without macro systems and helps implementers
build more elegant, expressive and efficient microKanrens in their
chosen host languages.

\begin{acks}

  Thanks to Ken Shan and Jeremy Siek, for helpful discussions and
  debates during design decision deliberations. Thanks also to Greg
  Rosenblatt and Michael Ballantyne for their insights and
  suggestions. We would also like to thank our anonymous reviewers
  for their insightful contributions.

\end{acks}

\printbibliography{}

\end{document}


%%% Local Variables:
%%% mode: latex
%%% TeX-master: t
%%% End:
