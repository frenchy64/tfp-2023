\documentclass[sigplan,draft,balance,pbalance,natbib=false]{acmart}
%% ∞
%% …
%% ₀
%% ₁
%% ₂
%% ₃
\usepackage{fontspec,newunicodechar}
\usepackage{hyphenat} %% electromagnetic\hyp{}endioscopy
\usepackage[shortcuts]{extdash}
\usepackage{microtype} %% provides tighter text formatting
\usepackage{flushend}
\usepackage{balance}
\usepackage{polyglossia} %% for xelatex use polyglossia, needed for csquotes
\setdefaultlanguage[variant=american]{english}

\PassOptionsToPackage{cache=false,newfloat=true}{minted}
\PassOptionsToPackage{unicode}{hyperref}

\usepackage[capitalize,nameinlink]{cleveref} %% adds \crefrange{}{} and \cpagerefrange{}
%% \BibTeX command to typeset BibTeX logo in the docs
\AtBeginDocument{%
  \providecommand\BibTeX{{%
    Bib\TeX}}}

\RequirePackage[
  datamodel=acmdatamodel,
%%  style=acmnumeric,
  ]{biblatex}

%% Declare bibliography sources (one \addbibresource command per source)
\addbibresource{mkw.bib}

\overfullrule=0pt

%\usepackage{nag} %% should complain about old and outdated commands
\usepackage{minted}
\usepackage[strict,autostyle]{csquotes} %% enquote &c that doesn't interfere w/cdlatex usage
%% http://ftp.lyx.org/pub/tex-archive/macros/latex/contrib/csquotes/csquotes.pdf
%% \usepackage[xspace]{ellipsis} they suggest using xetex instead
%% \usepackage{setspace} %% set the spacing (single double, etc)

\setmonofont{DejaVuSansMono}[Scale=MatchLowercase]
\tracinglostchars=2
\usepackage[defaultlines=3,all]{nowidow} %% Eliminates widow and orphans
\setlength {\marginparwidth }{2cm}
\usepackage{todonotes}

\newmintinline[rackinline]{racket}{}
\newmintinline{prolog}{}

\setcopyright{acmcopyright}
\copyrightyear{2023}
\acmYear{2023}
\acmDOI{XXXXXXX.XXXXXXX}

\acmConference[TFP '23]{Symposium on Trends in Functional
  Programming}{January 13-15, 2023}{Boston, Massachusetts}

\acmPrice{15.00}
\acmISBN{978-1-4503-XXXX-X/18/06}

\begin{document}

\title{Nearly Macro-free microKanren}

\author{Jason Hemann}
\email{jason.hemann@shu.edu}
\orcid{0000-0002-5405-2936}
\author{Daniel P. Friedman}
\email{dfried@cs.indiana.edu}
\orcid{0000-0001-9992-1675}

\renewcommand{\shortauthors}{Hemann et al.}

\begin{abstract}

  This paper describes changes to the microKanren implementation that
  make it more practical to use in a host language without macros.
  With the help of some modest runtime features common to most
  languages, we show how an implementer lacking macros can come closer
  to the expressive power that macros usually provide---with varying
  degrees of success. The result is a still functional microKanren that
  invites slightly shorter programs, and is relevant even to
  implementers that enjoy macro support. For those without it, we
  address some practical concerns that necessarily occur without
  macros so they can better weigh their options.

\end{abstract}

\begin{CCSXML}
<ccs2012>
   <concept>
       <concept_id>10011007.10011006.10011008.10011009.10011015</concept_id>
       <concept_desc>Software and its engineering~Constraint and logic languages</concept_desc>
       <concept_significance>500</concept_significance>
   </concept>
   <concept>
       <concept_id>10003752.10003790.10003795</concept_id>
       <concept_desc>Theory of computation~Constraint and logic programming</concept_desc>
       <concept_significance>300</concept_significance>
   </concept>
 </ccs2012>
\end{CCSXML}

\ccsdesc[500]{Software and its engineering~Constraint and logic languages}

%%
 %% Keywords. The author(s) should pick words that accurately describe
%% the work being presented. Separate the keywords with commas.
\keywords{logic programming, miniKanren, DSLs, embedding, macros}

\maketitle

\section{Introduction}

The authors designed microKanren~\cite{hemann2013muKanren} as a
compact relational programming language kernel to undergird a
miniKanren implementation. Macros are used to implement the
surrounding higher-level miniKanren operators and surface syntax.\@
microKanren is often used as a tool for understanding the guts of a
relational programming language through studying its implementation.
By re-implementing miniKanren as separate surface syntax macros over a
purely-function microKanren kernel, the authors hoped this separation
would simultaneously aid implementers when studying the source code,
and also that the functional core would make the language easier to
port to other functional hosts. To support both those efforts, they
also chose to program in a deliberately small and workaday set of
Scheme primitives.

The sum of those implementation restrictions, however, necessitates
some awkward compromises in places including binary logical operators,
one at a time local variable introduction, and leaks in the streams
abstractions. These made the surface syntax macros seem practically
mandatory, and fell short enough that we compromised on a purely
functional kernel in a pedagogical
exposition~\cite{friedman2018reasoned}. It also divided host languages
into the macro language \enquote{haves} and macro-less \enquote{have
  nots}. Here, we bridge some of that divide by re-implementing parts
of the kernel with some modest runtime features common to most
languages.

In this paper we:
%
\begin{itemize}

\item show how to functionally implement more general logical
  operators, cleanly obviating some of the surface macros

\item survey, the design space of purely functional implementation
  alternatives for the remaining macros in \emph{The Reasoned
    Schemer, 2nd Ed}'s~\cite{friedman2018reasoned} (\emph{TRS2e}) core
  language implementation, and weigh the trade offs and real-world
  consequences

\item suggest practical solutions for completely eliminating the
  macros in those places where the pure microKanren functional
  implementations had seemed impractical

\end{itemize}

This exercise resulted in some higher-level (variadic rather than just
binary) operators, a more succinct kernel language, and enabled some
performance improvement. Around half of the changes are applicable to
any microKanren implementation, and the more concise goal combinators
of~\cref{sec:conde} may also be of interest to implementers embedding
goal-oriented languages like Icon. The other half are necessarily
awkward yet practical strategies for those platforms lacking macro
support. Our re-implementation's source code for and the source for
our experimental results is available
at~\url{https://github.com/jasonhemann/tfp-2023/}.

In \cref{sec:all-aboard}, we illustrate by example what made surface
syntax macros feel practically mandatory. In \cref{sec:conde}, we
implement conjunction and disjunction, and in \cref{sec:impure} we
discuss the re-implementation of the impure operators. We discuss the
remaining macros in \cref{sec:functional}. We close some outstanding
questions on performance impacts of these implementation choices, and
consider how Kanren language implementers outside of the Scheme family
might benefit from these alternatives.

\section{All Aboard!}\label{sec:all-aboard}

We assume the reader is familiar with the miniKanren implementation of
\emph{The Reasoned Schemer, 2nd Ed} (\emph{TRS2e}). Although based on
microKanren, this implementation makes some concessions to efficiency
and safety and uses a few macros in the language kernel itself. In
addition to that implementation, in this paper we make occasional
references to earlier iterations such as
\citeauthor{hemann2016small}~\cite{hemann2016small}, an expanded
archival version of the 2013 paper~\cite{hemann2013muKanren}.

Haifa's Carmelit is the world's shortest subway system with a line of
just six stations; an example sufficiently small that modeling it
should be painless. But in microKanren, to model the order a passenger
visit the stops riding that subway end to end requires 11 logical
operator nodes, because microKanren only provides \emph{binary}
conjunctions and disjunctions. (\Cref{mnt:new-carmelit} contains this
paper's alternative solution, requiring just three.) For a logic
programming language, solely binary logical operators is too low
level. To our eyes, this makes the superficial syntax macros
practically mandatory, and host languages without a macro system are
out of luck.

Moreover, the microKanren language doesn't offer the programmer much
guidance in using that fine-grained control. For a series of $n$
goals, the programmer can associate them to the left, to the right, or
some mixtures of the two. The syntax does not obviously encourage any
one choice. Subtle changes in program structure can have profound
effects on performance, and mistakes are easy to make.

The \emph{TRS2e} microKanren's soft-cut operator, \rackinline{ifte},
is similarly low level. It permits a single test, a single consequent,
and a single alternative. To build an if-then-else cascade, a
microKanren programmer without the \rackinline{conda} surface macro
would need to code that unrolled cascade by hand.

The core \emph{TRS2e} language implementation relies on
macros \rackinline{fresh}, \rackinline{defrel}, and \rackinline{run}
to introduce new logic variables, globally define relations, and
execute queries. Earlier implementations of those same behaviors via
pure functional shallow embeddings, without macros, had some harsh
consequences. We will revisit those earlier implementations and their
trade-offs, survey the landscape of available choices, and suggest
performant compromises for those truly without macros, thus increasing
microKanren's \emph{practical} portability.

\section{\rackinline{disj} and \rackinline{conj} logical goal
  constructors}\label{sec:conde}

microKanren's binary \rackinline{disj₂} and \rackinline{conj₂}
operators are goal combinators: they each take two goals, and produce
a new goal. Disjunction and conjunction work slightly differently. A
\emph{goal} is an outcome the program attempts to achieve, a goal can
fail or succeed (and it can succeed many times). A goal executes with
respect to a \emph{state}, here the curried parameter \rackinline{s},
and the result is a \emph{stream} of states, and each entry is one
state that results from achieving that goal in the given state.
The
\rackinline{$append} function used in \rackinline{disj₂} is an
internal kernel primitive that combines two streams into one, with an
interleave mechanism to prevent starvation; the result is a stream of
the ways to achieve the two goals' disjunction.
The \rackinline{$append-map} function used in \rackinline{conj₂} is
to\rackinline{$append} what the standard \rackinline{append-map} is
to \rackinline{append}. The ways to achieve the conjunction of two
goals are all the ways to achieve the second goal from the state after
achieving the first goal.\@ \rackinline{$append-map} runs the second
goal over the stream of results from the first goal, and combines the
results of mapping together into a single stream representing the
conjunction of the two, again with special attention to interleaving
and starvation.

\begin{listing}
  \begin{minted}[autogobble,stripall]{racket}
(define ((disj₂ g₁ g₂) s)
  ($append (g₁ s) (g₂ s)))

(define ((conj₂ g₁ g₂) s)
  ($append-map g₂ (g₁ s)))
  \end{minted}
  \caption{microKanren \rackinline{disj₂} and \rackinline{conj₂}}
  \label{mnt:disj2-conj2}
\end{listing}

We want to implement disjunction and conjunction over arbitrary
quantities of goals, as functions. These implementations should
subsume the binary \rackinline{disj₂} and \rackinline{conj₂} and they
also should not use \rackinline{apply}. Further, they should not build
any extraneous closures: unnecessarily building closures at runtime is
always a bad idea. This re-implementation requires a host that
supports variable arity functions, a widely available feature included
in such languages as JavaScript, Ruby, Java, and Python. These
languages do not generally support macros and hence are beneficiaries
of this paper's contributions.

\begin{listing}
  \begin{minted}[autogobble,stripall]{racket}
  (define ((disj . gs) s)
    (cond
      ((null? gs) (list))
      (else (D ((car gs) s) (cdr gs) s))))

  (define (D s∞ gs s)
    (cond
      ((null? gs) s∞)
      (else
       (append∞ s∞
         (D ((car gs) s) (cdr gs) s)))))
  \end{minted}
  \caption{Eventual redefinition of \rackinline{disj}}
  \label{mnt:disj-reimplementation}
\end{listing}

\begin{listing}
  \begin{minted}[autogobble,stripall]{racket}
  (define ((conj . gs) s)
    (cond
      ((null? gs) (list s))
      (else (C (cdr gs) ((car gs) s)))))

  (define (C gs s∞)
    (cond
      ((null? gs) s∞)
      (else
       (C (cdr gs)
          (append-map∞ (car gs) s∞)))))
  \end{minted}
  \caption{Eventual redefinition of \rackinline{conj}}
  \label{mnt:conj-reimplementation}
\end{listing}

\Cref{mnt:disj-reimplementation,mnt:conj-reimplementation} show our
new implementations. We re-implement these operators as shallow
wrappers over simple folds. In each, the first steps are to dispense
with the trivial case, and then to call a recursive help function that
makes no use of variadic functions. That is, all of our focus will be
on the recurring over the list \rackinline{gs}. Unlike \rackinline{D},
the function \rackinline{C} does not take in the state \rackinline{s};
the help procedure does not need the state for conjunction. In each
recursive call, we accumulate by mapping (using that special delaying
implementation of \rackinline{append-map∞} for Kanren-language
streams) the next goal in the list. This left-fold implementation of
conjunction therefore left-associates the conjuncts.

\subsection{Deriving semantic equivalents}

A developer might derive these definitions as follows. We start with
the definition of a recursive \rackinline{disj} macro like one might
define as surface syntax over the microKanren \rackinline{disj₂}. As
this is not part of the microKanren language itself, we would like to
dispense with the macro and implement this behavior functionally. At
the cost of an \rackinline{apply}, we can build the corresponding
explicitly recursive \rackinline{disj} function.

\begin{listing}
\begin{minted}[autogobble,stripall]{racket}
(define-syntax disj
  (syntax-rules ()
    ((disj g) g)
    ((disj g₀ g₁ g ...)
     (disj₂ g₀ (disj g₁ g ...)))))

(define (disj g . gs)
  (cond
    ((null? gs) g)
    (else (disj₂ g (apply disj gs)))))
\end{minted}
  \caption{Deriving \rackinline{disj} function from macro}
  \label{mnt:disj-function-before-derivation}
\end{listing}

\noindent Since \rackinline{disj} produces and consumes goals, we can
η-expand the definition in \cref{mnt:disj-function-before-derivation}
by a curried parameter \rackinline{s}. We then split \rackinline{disj}
into two mutually-recursive procedures, to build the variant in
\cref{mnt:disj-function-split}.

\begin{listing}
\begin{minted}[autogobble,stripall]{racket}
(define ((disj g . gs) s)
  (D g gs s))

(define (D g gs s)
  (cond
    ((null? gs) (g s))
    (else ((disj₂ g (apply disj gs)) s))))
\end{minted}
  \caption{An η-expanded and split definition of \rackinline{disj}}
  \label{mnt:disj-function-split}
\end{listing}

\noindent We can replace the call to \rackinline{disj₂} in
\cref{mnt:disj-function-split} by its definition in terms of
\rackinline{append∞} and perform a trivial β-reduction. The explicit
\rackinline{s} argument suggests removing the call to
\rackinline{apply} and making \rackinline{D} recursive. The result is
the version of \rackinline{D} in \cref{mnt:disj-substituted-through}.
The definition of \rackinline{disj} remains unchanged from
\cref{mnt:disj-function-split}.

\begin{listing}
\begin{minted}[autogobble,stripall]{racket}
(define (D g gs s)
  (cond
    ((null? gs) (g s))
    (else
      (append∞ (g s)
        (D (car gs) (cdr gs) s)))))
\end{minted}
  \caption{Derivation of \rackinline{disj} function definition}
  \label{mnt:disj-substituted-through}
\end{listing}


\noindent In both clauses of \rackinline{D} we combine \rackinline{g}
and \rackinline{s}, this suggests constructing that stream
in \rackinline{disj} and passing it along. Adding the trivial base
case to that \rackinline{disj} yields the definition
in~\cref{mnt:disj-reimplementation}.

\begin{listing}
\begin{minted}[autogobble,stripall]{racket}
(define-syntax conj
  (syntax-rules ()
    ((conj g) g)
    ((conj g g₁ gs ...)
     (conj (conj₂ g g₁) gs ...))))

(define (conj g . gs)
  (cond
    ((null? gs) g)
    (else
     (apply conj
       (cons (conj₂ g (car gs)) (cdr gs))))))
\end{minted}
  \caption{\rackinline{conj₂}-based \rackinline{conj} macro and function}
  \label{mnt:conj-function-derived-definition}
\end{listing}

We can derive the definition of \rackinline{conj} from
\cref{mnt:conj-reimplementation} via a similar process. Starting with
the variadic function based on the macro in
\cref{mnt:conj-function-derived-definition}, we first η-expand and
split the definition.

\begin{listing}
\begin{minted}[autogobble,stripall]{racket}
  (define ((conj g . gs) s)
    (C g gs s))

  (define (C g gs s)
    (cond
      ((null? gs) (g s))
      (else
       ((apply conj
          (cons (conj₂ g (car gs)) (cdr gs)))
        s))))
\end{minted}
  \caption{Derivation of split \rackinline{conj} function definition}
  \label{mnt:conj-substituted-through}
\end{listing}
\noindent We next substitute for the definitions of \rackinline{conj} and
\rackinline{conj₂}.

\begin{listing}
\begin{minted}[autogobble,stripall]{racket}
  (define (C g gs s)
    (cond
      ((null? gs) (g s))
      (else
       (C (λ (s) (append-map∞ (car gs) (g s)))
          (cdr gs)
          s))))
\end{minted}
  \caption{Replacing \rackinline{apply} in \rackinline{C} function definition}
  \label{mnt:C-substituted-through}
\end{listing}

Finally, since \rackinline{C} only needs \rackinline{s} to
\emph{build} the stream, we can assemble the stream on the way
in---instead of passing in \rackinline{g} and \rackinline{s}
separately, we pass in their combination as a stream. The function is
tail recursive, we can change the signature in the one and only
external call and the recursive call. The result, after adding the
trivial base case to \rackinline{conj}, is shown in
\cref{mnt:conj-reimplementation}.

Both the functional and the macro based versions of
\cref{mnt:conj-function-derived-definition} use a left fold over the
goals, whereas the versions of \rackinline{disj} use a right fold.
This is not an accident.
%
% Folklore suggests left associating conjunctions tends to improve the
% performance of miniKanren's interleaving search. The authors know of
% no thorough algorithmic proof of such claims, but see for instance
% discussions and implementation in~\cite{rosenblatt2019first} for some
% of the related work so far. We have generally, however, resorted to
% small step visualizations of the search tree to explain the
% performance impact. The authors believe it is worth considering if we
% can make an equally compelling argument for this preference through
% equational reasoning and comparing the implementations of functions.
%
Folklore suggests that left associating conjunctions tends to improve
the performance of miniKanren's interleaving search. The authors know
of no thorough algorithmic proof of such claims, but see for instance
discussions and implementation in~\cite{rosenblatt2019first} for some
of the related work so far. In \cref{tab:???}, we display the results
of some micro benchmarks that suggest the same. We have generally,
however, resorted to small step visualizations of the search tree to
explain the performance impact. The authors believe it is worth
considering if we can make an equally compelling argument for this
preference through equational reasoning and comparing the
implementations of functions.

The benefits of a left-fold over conjunctions becomes a little more
obvious by comparison to a right-fold implementation after we
η-expand, unfold to a recursive help function, substitute in the
definition of \rackinline{conj₂}, and β-reduce. From there, we cannot
(easily) replace the \rackinline{apply} call by a recursive call
to \rackinline{C}, because we are still waiting for an \rackinline{s}.
We can only abstract over \rackinline{s} and wait; we show the upshot
of this sequence in \Cref{mnt:conj-right-fold-definition}. Since we
know that any call to \rackinline{append-map∞} we construct will
always yield a result, the version in \cref{mnt:conj-reimplementation}
is tail recursive. The equivalent right-fold implementation needs to
somehow construct a closure for every recursive call. Basic
programming horse sense suggests the more elegant variant from
\cref{mnt:conj-reimplementation}.

\begin{listing}
\begin{minted}[autogobble,stripall]{racket}
  (define ((conj g . gs) s)
    (C gs (g s)))

  (define (C gs s∞)
    (cond
      ((null? gs) s∞)
      (else
       (append-map∞
         (λ (s) (C (cdr gs) ((car gs) s)))
         s∞))))
\end{minted}
  \caption{A right-fold variant of \rackinline{conj} after some derivations}
  \label{mnt:conj-right-fold-definition}
\end{listing}

The new \rackinline{disj} and \rackinline{conj} functions are, we
believe, sufficiently high-level for programmers in implementations
without macros. Though this note mainly concerns working towards an
internal macro-less kernel language, it may also have something to say
about the miniKanren-level surface syntax, namely that even the
miniKanren language could do without its \rackinline{conde} syntax (a
disjunction of conjunctions that looks superficially like
Scheme's \rackinline{cond}) and have the programmer use these new
underlying logical primitives. We
implement \rackinline{carmelit-subway} as an example in
\cref{mnt:new-carmelit}, and it looks far cry better than the 11
binary logical operator nodes the programmer would have needed to
write in a microKanren language without macros.

\begin{listing}[h]
  \begin{minted}[autogobble,stripall]{racket}
(defrel (carmelit-subway a b c d e f)
  (disj
    (conj (== a 'carmel-center)
          (== b 'golomb)
          (== c 'masada)
          (== d 'haneviim)
          (== e 'hadar-city-hall)
          (== f 'downtown))
    (conj (== a 'downtown)
          (== b 'hadar-city-hall)
          (== c 'haneviim)
          (== d 'masada)
          (== e 'golomb)
          (== f 'carmel-center))))
  \end{minted}
  \caption{A new Carmelit subway without \rackinline{conde}}
  \label{mnt:new-carmelit}
\end{listing}

\section{Tidying up the Impure Operators}\label{sec:impure}

The miniKanren \rackinline{conda} operator that provides nested
\enquote{if-then-else} behavior relies on the
microKanren \rackinline{ifte} underlying it. The definition
of \rackinline{conda} (see \cref{mnt:conda-implementation}) requires
one or more conjuncts per clause and one or more clauses. The last
line of \rackinline{conda} contains the only place in the
implementation that relies structurally on permitting nullary
conjunctions, or disjunctions, of goals. Everywhere else conjunctions
are one-or-more, and this one structural dependency is off-putting.
Having already broached the topic of changing surface syntax, we
mention a temptation to rewrite the second pattern in
miniKanren's \rackinline{conda} to demand \emph{two} or more goals in
each if-then clause and removing the dependency.

\begin{listing}
  \begin{minted}[autogobble,stripall]{racket}
(define-syntax conda
  (syntax-rules ()
    ((conda (g₀ g ...)) (conj g₀ g ...))
    ((conda (g₀ g ...) ln ...)
     (ifte g₀ (conj g ...) (conda ln ...)))))

(define ((ifte g₁ g₂ g₃) s)
  (let loop ((s∞ (g₁ s)))
    (cond
      ((null? s∞) (g₃ s))
      ((pair? s∞)
       (append-map∞ g₂ s∞))
      (else (lambda ()
              (loop (s∞)))))))
  \end{minted}
  \caption{A typical implementation of \rackinline{conda}}
  \label{mnt:conda-implementation}
\end{listing}

Some microKanren programmers without macros would be perfectly
satisfied just using \rackinline{ifte} directly, especially so given
the research community's focus on purely relational programming. But
just as the standard forked \rackinline{if} begat
McCarthy's \rackinline{if} notation and \rackinline{cond}, a
programmer may eventually feel the need for a nested implementation.
Here is a functional implementation of that cascade behavior.

\begin{listing}
  \begin{minted}[autogobble,stripall]{racket}
(define ((conda q a . q-and-a*) s)
  (A (q s) a q-and-a* s))

(define (A s∞ a q-and-a* s)
  (cond
    ((null? s∞)
     (cond
       ((null? (cdr q-and-a*)) ((car q-and-a*) s))
       (else (A ((car q-and-a*) s)
                (cadr q-and-a*)
                (cddr q-and-a*)
                s))))
    ((pair? s∞) (append-map∞ a s∞))
    (else (lambda () (A (s∞) a q-and-a* s)))))
  \end{minted}
  \caption{A functional \rackinline{conda} implementation}
  \label{mnt:conda-good-re-implementation}
\end{listing}

The implementation in \cref{mnt:conda-good-re-implementation} includes
the delay-and-restart behavior of \rackinline{ifte} together
with \rackinline{conda}'s logical cascade. The \rackinline{s∞} can be
either empty, non-empty, or a function of no arguments. In the last
case, we invoke \rackinline{s∞}. Rather than building a largely
redundant implementation of \rackinline{condu}, we expose the
higher-order goal \rackinline{once} to the user. The definition
of \rackinline{once} in \cref{mnt:condu-reimplementation} is taken
directly from~\cite{friedman2018reasoned}. The programmer can
simulate \rackinline{condu} by wrapping \rackinline{once} around every
test goal.

\begin{listing}
  \begin{minted}[autogobble,stripall]{racket}
(define (once g)
  (lambda (s)
    (let loop ((s-inf (g s)))
      (cond
        ((null? s-inf) '())
        ((pair? s-inf)
         (cons (car s-inf) '()))
        (else (lambda ()
                (loop (s-inf))))))))
  \end{minted}
  \caption{The \rackinline{once} function} %% \rackinline{}
  \label{mnt:condu-reimplementation}
\end{listing}


\section{Remainders and Practicalities}\label{sec:functional}

We have not fully obviated the use of macros. In this section we
collect together some workarounds to obviate macros in the rest of the
implementation. Some of these come with significant drawbacks. With
these, however, a programmer in even a pedestrian functional language
should be able to directly translate the implementation and our test
programs.

\paragraph{\rackinline{define}}

The microKanren programmer can just use their host
language's \rackinline{define} feature to construct relations as
host-language functions, and manually introduce the delays in
relations. This may be a larger concession than it looks, since it
exposes the delay and interleave mechanism to the user, and both
correct interleaving and even the termination of relation
\emph{definitions} rely on a whole-program correctness property of
relation definitions having a delay. \Cref{mnt:subtleo} relies on a
help function \rackinline{Zzz} to introduce delays, akin to some
earlier implementations~\cite{hemann2013muKanren}. Another downside
worth mentioning is the programmer must now take extra care not to
provide multiple goals to define. The \rackinline{define} form will
treat all but the last expression as statements and silently drop
them, rather than conjoin them as in \rackinline{defrel}. That small
\emph{gotcha} can be subtle but significant drawback.

\begin{listing}
  \begin{minted}[autogobble,stripall]{racket}
(define (subtleo x)
  (Zzz
    (disj
      (subtleo x)
      (== x 'cat))))
  \end{minted}
  \caption{Omitting the delay is a subtle bug}
  \label{mnt:subtleo}
\end{listing}

\paragraph{\rackinline{fresh}}

In any implementation there must be some mechanism to produce the next
fresh variable. For example, we treat the natural numbers as an
indexed set of variables, and we thread the current index through the
computation. We use \rackinline{add1} to get the next index; to go
from index to variable is the identity function. For another example,
we could represent each variable using a unique memory location,
sokuza-kanren~\cite{kiselyov2006taste} style, and the operation to
produce a new variable requires introducing an unused memory location.
Depending on the implementation of variables, you may also need
additional functions to support your implementation of variables. If
variables are not from an indexed set, you may also need an operation
to (re) construct specifically the first element of the set, or
otherwise store that value for later re-use.

Of course, one of these approaches requires memory allocation and
external global state, while the other does not. Furthermore, the
latter approach models logic variables as coming from a single global
pool rather than reusing them separately across each disjunct, and so
requires some global store and strictly more logic variables overall.

With this latter approach, however, we can expose \rackinline{var}
directly to the programmer and the programmer can use \rackinline{let}
bindings to introduce several logic variables simultaneously.

\paragraph{\rackinline{run}}

We note that we can also implement \rackinline{run} without using
macros. Using the purely-functional implementation of logic variables,
the definitions of \rackinline{run} and \rackinline{run*} easily
translate to functions
like \rackinline{call/initial-state}~\cite{hemann2013muKanren}. The
query is itself expressed as a goal that introduces the first logic
variable \rackinline{q}. The pointer-based logic variable approach
forces the programmer to explicitly invoke \rackinline{reify} as
though it were a goal as the last step of executing the query, as in
\cref{mnt:run-query}.

\begin{listing}
  \begin{minted}[autogobble,stripall]{racket}
(call/initial-state 1
  (let ((q (var 'q)))
    (conj
      (let ((x (var 'x)))
        (== q x))
      (reify q))))
  \end{minted}
  \caption{Queries as expressed with global-state variables}
  \label{mnt:run-query}
\end{listing}

\section{Future Work}\label{sec:conclusion}

This note shows how to provide a somewhat more concise core language
that significantly reduces the need for macros. The result almost
rivals the expressivity of the full \enquote{microKanren + macros}
approach. Variadic functions make this implementation much more
convenient for the end programmer, and Scheme's polyvariadic function
syntax ensures at a host-language level that the microKanren
programmer provides at least one parameter to \rackinline{conj}
and \rackinline{disj}.

The old desugaring macros do not seem to suggest how to associate the
calls to the binary primitives---both left and right look equally
nice. Forcing ourselves to program the solution functionally, and the
restrictions we placed on ourselves in this reimplementation, removed
a degree of implementation freedom and led us to what seems like the
right solution.

The result is closer to the design of Prolog, where the user
represents conjunction of goals in the body of a clause with a comma
and disjunction, either implicitly in listing various clauses or
explicitly with a semicolon. We assume it is agreed that our
definitions of \rackinline{disj} and \rackinline{conj} themselves are
sufficiently high-level operators for a surface language and that the
zero-element base cases are at best unnecessary and likely
undesirable; given the opportunity to define a surface language and
its desugaring, we really shouldn't tempt the programmer by making
undesirable programs representable when we can avoid it.

Techniques for
implementing \rackinline{defrel}, \rackinline{fresh} and
\rackinline{run} (and \rackinline{run*}) without macros come with
serious drawbacks. These include exposing the implementation of
streams and delays, and the inefficiency and clumsiness of introducing
variables one at a time, or the need to reason with global state.

From time to time we find that the usual miniKanren implementation is
\emph{itself} lower-level than we would like to program with
relations. Early microKanren implementations restrict themselves to
\rackinline{syntax-rules} macros. Some programmers use macros to
extend the language further as with
\rackinline{matche}~\cite{keep2009pattern}. Some constructions over
miniKanren, such as
\rackinline{minikanren-ee}~\cite{ballantyne2020macros}, may rely on
more expressive macro systems like
\rackinline{syntax-parse}~\cite{culpepper2012fortifying}.

We would still like to know if our desiderata here are \emph{causally}
related to good miniKanren performance. Can we reason at the
implementation level and peer through to the implications for
performance? If left associating \rackinline{conj} is indeed uniformly
a dramatic improvement, the community might consider reclassifying
left-associative conjunction as a matter of correctness rather than an
optimization, as in \enquote{tail call optimization} vs.
\enquote{Properly Implemented Tail Call
  Handling}~\cite{felleisen2014requestions}. Regardless, we hope this
document helps narrow the gap between implementations in functional
host languages with and without macro systems and helps implementers
build more elegant, expressive and efficient Kanrens in their chosen
host languages.

\begin{acks}

  Thanks to Ken Shan and Jeremy Siek, for helpful discussions and
  debates during design decision deliberations. Thanks also to Greg
  Rosenblatt and Michael Ballantyne for their insights and
  suggestions. We would also like to thank our anonymous reviewers
  for their insightful contributions.

\end{acks}

\printbibliography{}

\end{document}


%%% Local Variables:
%%% mode: latex
%%% TeX-master: t
%%% End:
